%% Copernicus Publications Manuscript Preparation Template for LaTeX Submissions
%% ---------------------------------
%% This template should be used for copernicus.cls
%% The class file and some style files are bundled in the Copernicus Latex Package, which can be downloaded from the different journal webpages.
%% For further assistance please contact Copernicus Publications at: production@copernicus.org
%% https://publications.copernicus.org/for_authors/manuscript_preparation.html


%% Please use the following documentclass and journal abbreviations for preprints and final revised papers.

%%2-column papers and preprints
\documentclass[journal abbreviation, manuscript]{copernicus}



%% Journal abbreviations (please use the same for preprints and final revised papers)


% Advances in Geosciences (adgeo)
% Advances in Radio Science (ars)
% Advances in Science and Research (asr)
% Advances in Statistical Climatology, Meteorology and Oceanography (ascmo)
% Annales Geophysicae (angeo)
% Archives Animal Breeding (aab)
% ASTRA Proceedings (ap)
% Atmospheric Chemistry and Physics (acp)
% Atmospheric Measurement Techniques (amt)
% Biogeosciences (bg)
% Climate of the Past (cp)
% DEUQUA Special Publications (deuquasp)
% Drinking Water Engineering and Science (dwes)
% Earth Surface Dynamics (esurf)
% Earth System Dynamics (esd)
% Earth System Science Data (essd)
% E&G Quaternary Science Journal (egqsj)
% European Journal of Mineralogy (ejm)
% Fossil Record (fr)
% Geochronology (gchron)
% Geographica Helvetica (gh)
% Geoscience Communication (gc)
% Geoscientific Instrumentation, Methods and Data Systems (gi)
% Geoscientific Model Development (gmd)
% History of Geo- and Space Sciences (hgss)
% Hydrology and Earth System Sciences (hess)
% Journal of Bone and Joint Infection (jbji)
% Journal of Micropalaeontology (jm)
% Journal of Sensors and Sensor Systems (jsss)
% Magnetic Resonance (mr)
% Mechanical Sciences (ms)
% Natural Hazards and Earth System Sciences (nhess)
% Nonlinear Processes in Geophysics (npg)
% Ocean Science (os)
% Primate Biology (pb)
% Proceedings of the International Association of Hydrological Sciences (piahs)
% Scientific Drilling (sd)
% SOIL (soil)
% Solid Earth (se)
% The Cryosphere (tc)
% Weather and Climate Dynamics (wcd)
% Web Ecology (we)
% Wind Energy Science (wes)


%% \usepackage commands included in the copernicus.cls:
%\usepackage[german, english]{babel}
%\usepackage{tabularx}
%\usepackage{cancel}
%\usepackage{multirow}
%\usepackage{supertabular}
%\usepackage{algorithmic}
%\usepackage{algorithm}
%\usepackage{amsthm}
%\usepackage{float}
%\usepackage{subfig}
%\usepackage{rotating}



\begin{document}


\title{DisReserv v1.0: Analytical displacement solution due to reservoir compaction with arbitrary geometry and under arbitrary pressure changes}


% \Author[affil]{given_name}{surname}

\Author[1]{Valeria}{C. F. Barbosa}
\Author[1]{Vanderlei}{C. Oliveira Jr.}
\Author[1,2]{Andre}{D. Arelaro}

\affil[1]{Department of Geophysics, Observat\'{o}rio Nacional, Rio de Janeiro, Brazil}
\affil[2]{PETROBRAS, Rio de Janeiro, Brazil}


%% The [] brackets identify the author with the corresponding affiliation. 1, 2, 3, etc. should be inserted.

%% If an author is deceased, please mark the respective author name(s) with a dagger, e.g. "\Author[2,$\dag$]{Anton}{Aman}", and add a further "\affil[$\dag$]{deceased, 1 July 2019}".

%% If authors contributed equally, please mark the respective author names with an asterisk, e.g. "\Author[2,*]{Anton}{Aman}" and "\Author[3,*]{Bradley}{Bman}" and add a further affiliation: "\affil[*]{These authors contributed equally to this work.}".


\correspondence{Valeria C. F. Barbosa (valcris@on.br)}

\runningtitle{Displacement due to reservoir compaction}

\runningauthor{Barbosa, V. C. F., Oliveira Jr., V. C. and Arelaro, A. D.}



\received{}
\pubdiscuss{} %% only important for two-stage journals
\revised{}
\accepted{}
\published{}

%% These dates will be inserted by Copernicus Publications during the typesetting process.


\firstpage{1}

\maketitle


%%\twocolumn
\begin{abstract}
We have presented analytical solutions for the displacement due to reservoir compaction with arbitrary geometry and under arbitrary pressure changes.
These solutions are based on the similarity between the gravitational potential yielded by  a volume source under a density variation and the displacement field yielded by a volume source in a half-space under a pressure variation.
This similarity enables the use of closed expressions of the gravitational potential and its derivatives for calculating the displacement field due to a volume source under a pressure variation.
We discretized the reservoir as a grid of 3D right rectangular prisms  juxtaposed in the horizontal and vertical directions.
Each grid prism has homogeneous pressure; however, pressure variations among different prisms are allowed. 
This parametrization of the reservoir yields a piecewise-constant distribution of pressure  in the subsurface. 
The discrete reservoir modeling to calculate the displacement field due to this pressure variation  follows the nucleus of strain concept in which the center of each prism represents the coordinate of a nucleus of strain.
The displacement due to a nucleus of strain is considered the infinitesimal element of the 
displacement due to an infinitesimal reservoir.
The displacement field due to the pressure of each prism is calculated by integrating the infinitesimal element of the displacement over the volume of the prism.
Finally, the displacement due to the reservoir can be approximated by the sum of the contributions of each prism of the discretized model.
We provide python codes (DisReserv) to calculate the displacement fields due to a reservoir with arbitrary shape and distribution of pressure changes. 
The displacement field is calculated inside and outside the reservoir. 
We calculate the displacement fields in three scenarios: i) cylindrical reservoir under uniform depletion; ii) cylindrical reservoir under non uniform depletion and iii) reservoir with arbitrary geometry and under arbitrary pressure distribution.
By calculating the  stress field at the free surface, we verify that the zero stress condition is satisfied.
\end{abstract}


\copyrightstatement{TEXT}


\introduction  %% \introduction[modified heading if necessary]
The surface subsidence due to oil or gas withdrawal from a reservoir in the subsurface may occur as a result of geomechanical changes caused by pressure drop.
The phenomenon of subsidence by fluid extraction  has been observed in a variety of oil fields, e.g., the Ekofisk field, southern North Sea \citep{Borges2020} and the Groningen gas field in the northeast Netherlands \citep{vanThienen-visser&Fokker17}.
Because the subsidence close to hydrocarbon fields under production can induce earthquakes (e.g., \citeauthor{Dahmetal15}, \citeyear{Dahmetal15}; \citeauthor{Grigolietal17}, \citeyear{Grigolietal17}), the petroleum companies  have been an increased interest in monitoring the magnitude and distribution of subsidence resulting from reservoir depletion.
The subsidence monitoring is accomplished by means of calculating the displacement field for a given set of  reservoir properties.

The possibility of occurrence of catastrophic events, related to subsidence due to extraction or injection of fluids in reservoir, has stimulated efforts to  develop analytical methods for modeling the displacement and stress fields due to reservoir compactation.
The physical foundation of the displacement, stress and strain fields in the subsurface due to a reduction of pressure in the reservoir comes from the theory of thermoelasticity.
Theory of thermoelasticity has been laid in the first half of the nineteenth century to describe the interaction between the thermal field and elastic bodies.
In the uncoupled thermoelasticity theory for quasi-static problems (i.e., problems with negligible inertia effects), \cite{Goodier37} employed the method of superposition using
displacement potential functions and introduced the  concept of nucleus of thermoelastic strain in an infinite space.
Specifically, Goodier’s \citeyear{Goodier37} method simplified the thermoelastic problem by replacing it by an isothermal elastic problem with different boundary conditions together with the solution of a Poisson’s equation \citep{Tao71}. 
\cite{Mindlin&Cheng50} and \cite{Sen51} extended the Goodier's method to a homogenous half-space.
\cite{Sharma56} deduced the displacement and stress fields in an infinite elastic plate due to a nucleus of thermoelastic strain located at a point inside it by using infinite integrals involving Bessel functions.

The subsidence resulting from reservoir depletion is in the context of poroelastic theory. 
\cite{Geertsma57} remarked the analogy between the theories of thermoelasticity and poroelasticity.
To our knowledge, \cite{Geertsma73} was the first to solve the poroelastic problem by using the nucleus-of-strain concept in the half-space, which in turn was proposed by \cite{Mindlin&Cheng50} and \cite{Sen51} in the theory of thermoelasticity.
\cite{Geertsma73} derived analytical expressions for the stress and displacement fields for a thin disk-shaped reservoir. 
\cite{Segall92} followed \cite{Geertsma73} and extended the analytical solutions of the displacement and stress fields assuming general axisymmetric geometries and an arbitrary radial pressure distribution. 

\cite{Geertsma&Opstal73} applied the nucleus of strain concept in the half-space 
to calculate the spatial subsidence distribution due to the production of reservoir with an arbitrary 3D shape.
By assuming a producing reservoir embedded in an a homogeneous, isotropic, and elastic medium, and a reservoir model in which the pressure perturbations are related to the displacement field by a linear relationship, \cite{Geertsma&Opstal73} discretized the reservoir into a grid of  prisms and calculated the displacement due to the pressure change in the whole reservoir  by the superposition of the displacement due to the constant pressure change in each prism.
\cite{Tempone10} adopted the same reservoir model used in \cite{Geertsma&Opstal73} and 
extended the nucleus of strain concept in the half-space to consider the effects of a rigid basement. 
Similarly, \cite{Tempone10} assumed a reservoir embedded in an a homogeneous, isotropic, and elastic medium and calculated the displacement, stress and strain fields subject to uniform depletion.
The main drawbacks in \cite{Geertsma&Opstal73} and \cite{Tempone10} are the assumption of homogeneous reservoir and the solution is only valid outside the reservoir.
In these case, the displacements within the reservoir are calculated by a linear interpolation of the displacements at the upper and lower edges of the reservoir \citep{Tempone12}.

Considering an inhomogeneous poroelastic model consists of layered stratigraphy, \cite{Mehrabian&Abousleiman15} developed closed-form formulae for the  displacement and stress  fields outside and inside of the reservoir embedded within elastic strata with different mechanical properties and subjected to pore pressure disturbances due to fluid extraction or injection. 
By assuming a linear elastic semi-infinite medium, \cite{Munoz&Roehl17} developed analytical solution for the  displacement field  outside and inside of an arbitrarily-shaped reservoir under arbitrary distribution of pressure changes. 
\cite{Munoz&Roehl17} parametrized the reservoir into a grid of 3D prisms and used the nucleus of strain concept. 
The nucleus of strain is taking as an infinitesimal volume element  for each prism and the displacement solution due to each prism is obtained by a three-dimensional integration over the prism volume. 
The displacement field outside and inside of the reservoir is given by the summation of the displacement fields of the displacements produced by all prisms setting up the reservoir model.
 
The present work assumes a linear elastic semi-infinite medium and provides an analytical solution for displacement field due to an arbitrarily-shaped reservoir under arbitrary distribution of pressure changes. 
Like \cite{Munoz&Roehl17} we used the nucleus of strain concept and discretized the
reservoir into a grid of 3D prisms along the $x-$, $y-$ and $-z$directions.
We also consider the nucleus of strain as an infinitesimal volume element  for each prism and the displacement solution due to each prism is obtained by a three-dimensional integration over the prism volume. 
The final displacement field due to the whole reservoir is the sum of the displacements produced by the prisms. 
In contrast with \cite{Munoz&Roehl17}'s method we take advantage the similarity between 
the  equations for calculating the displacement field due to a volume source in a half-space under a pressure variation and the gravitational potential due to a volume source under a density variation.
This similarity makes possible the use of closed expressions of the gravitational potential and its derivatives produced by the 3D right rectangular prism derived by \cite{Nagyetal2000} and (\citeyear{Nagyetal2002}) and \cite{Fukushima2020}  for calculating the displacement field due to a volume source under a pressure variation.
The adopted exact analytical formulae of the gravitational field  are valid expressions either outside or inside the prisms because the implemented expressions make use of  modified arctangent function proposed by \cite{Fukushima2020}. 
We present routines written in Python language (Python 3.7.6)  to calculate the displacement fields due to a reservoir with arbitrary shape and distribution of pressure changes. 
We validate our equations by verifying that at the free surface the stress fields are null.
Tests with synthetic data validate our approach. 

\section{THEORY}

The subsidence or displacement, stress and strain fields in the subsurface caused by reservoir compactation due to hydrocarbon production are grounded on the theory of thermoelasticity. 

The Goodier’s thermoelastic displacement potential $\phi$ satifies the Poisson's equation \citep{Goodier37}, i.e.:
\begin{equation}
\nabla^{2} \phi =  m \: T,
\label{eq:poisson}
\end{equation}
where $\nabla^{2}$ is the Laplacian operator, $T $ is the temperature difference and 
\begin{equation}
m =  \:  \alpha \: \frac{1 + \nu}{ 1 -\nu},
\label{eq:m}
\end{equation}
where $\alpha$  is the coefficient of linear thermal expansion and $\nu$ is the Poisson's ratio.

From the potential theory, a particular  solution of equation \ref{eq:poisson} is
\begin{equation}
\phi(x,y,z) = -  \frac{m}{4 \pi} \int\int\limits_{v}\int \frac{T(x^{\prime}, y^{\prime}, z^{\prime} )}{\sqrt{(x - x^{\prime})^{2} + (y - y^{\prime})^{2} + (z - z^{\prime})^{2}}} \: dv,
\label{eq:phi}
\end{equation}
where $\phi(x,y,z)$ represents the Newtonian gravitational potential \citep{Kellogg29} at the coordinates $x, y$ and $z$ produced by a continuous distribution of mass defined by a set of very small masses $ -  \frac{m}{4 \pi}  \: T( x^{\prime}, y^{\prime}, z^{\prime} ) \: dv$, where $T(x^{\prime}, y^{\prime}, z^{\prime}) $ stands for the density distribution. 
The integral in equation \ref{eq:phi} is conducted over the coordinates $x^{\prime}$,$y^{\prime}$ and $z^{\prime}$, denoting, respectively, the x-, y-, and z-coordinates of an arbitrary point belonging to the interior of the volume $v$ of the gravity source. 


From equation \ref{eq:phi} and the potential-field theory, \citet{Goodier37} showed that if an element of volume $dv$ in the infinite solid is at a temperature $T(x^{\prime}, y^{\prime}, z^{\prime}) $, the remainder being at temperature zero, the displacement vector $\bf{u}$ caused by this temperature is the gradient of the Goodier’s thermoelastic displacement potential, i.e.,
\begin{equation}
{\bf{u}} = {\bf {\nabla}} \phi(x,y,z). 
\label{eq:displacement-general}
\end{equation}
where $\bf{\nabla}$ is the gradient operator.  

To a homogenous half-space, \citet{Mindlin&Cheng50} showed that the method proposed by  \citet{Goodier37}  can be extended by the displacement solution given by:
\begin{equation}
\bf{u} = {\bf{\nabla}} \: \phi_{1} \: + \: {\bf{{\nabla}_{2}}} \phi_{2}, 
\label{eq:displacement-solution}
\end{equation}
where $\phi_{1} \equiv \phi_{1}(x,y,z)$ is the potential defined in equation \ref{eq:phi}, 
$\phi_{2} \equiv  \phi_{2}(x,y,z)$
is defined as "image potential" \citep{Segall92} due to a image point at the coordinates
$(x^{\prime}, y^{\prime}, -z^{\prime} )$ and the operator $\bf{{\nabla}_{2}}$ is expressed by
\begin{equation}
{\bf{{\nabla}_{2}}} = (3 - 4\nu){\bf{\nabla}} + 2 {\bf{\nabla}} z \frac{\partial }{\partial z}  - 4(1- \nu){\bf{\hat{z}}} {\bf{{\nabla}^{2}_{z}}},
\label{eq:nabla2}
\end{equation}
where ${\bf{\hat{z}}}$ is the unit vector in the $z-$direction and 
${\bf{{\nabla}^{2}_{z}}}$ is a scalar operator in which the operand is firtly multiplied by z and then operated upon by Laplacian operator ${\bf{\nabla}^{2}}$. 

Equation \ref{eq:displacement-solution} is the displacement solution for the variation of temperature due to a single nucleus of strain buried at depth $z^{\prime}$ in a
semi-infinite homogeneous medium. 
In the right hand side of equation \ref{eq:displacement-solution}, the first term 
$ {\bf{\nabla}} \: \phi_{1} $ represents the displacement in an infinite medium, and the second term represents a correction of the displacement due a half-space, also known as image nucleus solution.

\section{METHODOLOGY}

Let's assume that a reservoir in the interior of the Earth is subject to a compactation due to hydrocarbon production. 
The compactation is caused by the pressure change within the reservoir, which in turn causes  a surface subsidence (or surface displacement). 
We discretized the reservoir into an $ m_{x} \times m_{y} \times m_{z} $ grid of 3D vertical juxtaposed prisms $(m_{x} \cdot m_{y} \cdot m_{z} = M)$ in which
the pressure within each prism is assumed to be constant and known. 
Each grid prism in the reservoir model may undergo a distinct pressure change.
Hence, the subsidence effect is the displacement field due to the pressure change throughout the reservoir and it can be calculated by the sum of the displacement produced by each prism.


The discrete forward modeling to calculate the displacement and stress fields due to a piecewise-constant distribution of the pressure contrast within a reservoir follows the nucleus of strain approach. 
By assuming that the center of each prism represents the coordinate of a nucleus of
strain, we can calculate the displacement field due to the pressure contrast of this prism by integrating over its volume.
The displacement solution for a single nucleus of strain in a homogeneous elastic semi-infinite medium (equation \ref{eq:displacement-solution}) will be used as an element of the displacement.

\subsection{The discrete forward modeling due to a nucleus of strain in a homogeneous elastic semi-infinite medium}\label{solution-nucleus}

Here, we use a Cartesian coordinate system with the $x-$axis pointing to north, the $y-$axis pointing to east and the $z-$axis pointing downward.
By considering the discrete form of equation \ref{eq:displacement-solution}, the displacement field  $ {\bf{u}}_{i} \equiv  {\bf{u}}(x_{i}, y_{i}, z_{i})$ at an arbitrary point $(x_{i}, y_{i}, z_{i})$ due to the $j$th nucleus of strain at the coordinates 
$(x^{\prime}_{j}, y^{\prime}_{j}, z^{\prime}_{j})$ will be calculated by
\begin{equation}
{\bf{u}}_{i}  = {\bf{\nabla}} \: 
{\phi_{1}(x_{i}, y_{i}, z_{i}, x^{\prime}_{j}, y^{\prime}_{j}, z^{\prime}_{j})} \: + \: {\bf{{\nabla}_{2}}} {\phi_{2}(x_{i}, y_{i}, z_{i}, x^{\prime}_{j}, y^{\prime}_{j}, z^{\prime}_{j})}, 
\label{eq:displacement_ui}
\end{equation}
In equation \ref{eq:displacement_ui}, the functions 
$\phi_{1} \equiv {\phi_{1}(x_{i}, y_{i}, z_{i}, x^{\prime}_{j}, y^{\prime}_{j}, z^{\prime}_{j})}$ and  
$\phi_{2} \equiv {\phi_{1}(x_{i}, y_{i}, z_{i}, x^{\prime}_{j}, y^{\prime}_{j}, z^{\prime}_{j})}$
are, respectively, given by 
\begin{equation}
\phi_{1} = - \frac{C_m}{4 \pi}  \frac{\Delta p_{j} \: \:dv_j}{ {R_1}_{ij} }
\label{eq:nucleus-phi1}
\end{equation}
and
\begin{equation}
\phi_{2} = - \frac{C_m}{4 \pi}  \frac{\Delta p_{j} \: \:dv_j}{ {R_2}_{ij} }.
\label{eq:nucleus-phi2}
\end{equation}

In equations \ref{eq:nucleus-phi1} and \ref{eq:nucleus-phi2}, $\Delta p_j$  is the pressure contrast of the $j$th nucleus, $dv_j$ is an infinitesimal element of volume of the $j$th nucleus, and $C_m$ is the uniaxial compaction coefficient (see \citeauthor{Geertsma66}, \citeyear{Geertsma66}; \citeauthor{Tempone10}, \citeyear{Tempone10} and \citeauthor{Munoz&Roehl17}, \citeyear{Munoz&Roehl17}) given by
\begin{equation}
C_m = \frac{1}{E} \: \frac{(1 + \nu) (1  - 2\nu)}{(1-\nu)},
\label{eq:Cm}
\end{equation}
where $E$ is the Young’s modulus.

In equation \ref{eq:nucleus-phi1}, $ {R_1}_{ij}$ is the distance from the $i$th coordinate point of the displacement $ (x_{i}, y_{i}, z_{i})$ to the $j$th coordinate of the nucleus of strain $(x^{\prime}_{j}, y^{\prime}_{j}, z^{\prime}_{j})$, i.e.:
\begin{equation}
{R_1}_{ij} = {\sqrt{(x_{i}- x^{\prime}_{j})^{2} + (y_{i} - y^{\prime}_{j})^{2} + 
(z_{i} - z^{\prime}_{j})^{2}}}.
\label{eq:R1}
\end{equation}
In equation \ref{eq:nucleus-phi2}, $ {R_2}_{ij}$ is the distance from the $i$th coordinate point of the displacement $ (x_{i}, y_{i}, z_{i})$ to the $j$th coordinate of the image nucleus $(x^{\prime}_{j}, y^{\prime}_{j}, - z^{\prime}_{j})$, i.e.:
\begin{equation}
{R_2}_{ij} = {\sqrt{(x_{i}- x^{\prime}_{j})^{2} + (y_{i} - y^{\prime}_{j})^{2} + 
(z_{i} + z^{\prime}_{j})^{2}}}.
\label{eq:R2}
\end{equation}
Figure \ref{fig:nucleus_strain} shows a schematic representation of the geometry
of the nucleus of strain problem in a semi-infinite medium. 
The $j$th nucleus of strain is located at the coordinates 
$(x^{\prime}_{j}, y^{\prime}_{j}, z^{\prime}_{j})$. 
The $j$th image nucleus is located at the coordinates 
$(x^{\prime}_{j}, y^{\prime}_{j}, - z^{\prime}_{j})$.
The distances from the $i$th coordinate point of the displacement $ (x_{i}, y_{i}, z_{i})$ to the $j$th nucleus of strain and to the $j$th image nucleus are, respectively,  
$ {R_1}_{ij}$ (equation \ref{eq:R1})  and $ {R_2}_{ij}$ (equation \ref{eq:R2}) 
The free surface is a horizontal plane where the components of the stress are null.
 

\begin{figure}[h]
\includegraphics[width=8.3cm]{Fig/Figure_Nucleus_Strain.png}
\caption{Schematic representation of the geometry of the nucleus of strain in a semi-infinite medium. After \cite{Munoz&Roehl17}. 
The adopted Cartesian coordinate system considered the $x-$axis pointing to north, the $y-$axis pointing to east and the $z-$axis pointing downward.} 
\label{fig:nucleus_strain}
\end{figure}


Following the discrete form of the displacement solution (equation \ref{eq:displacement_ui}), 
the displacement field at the coordinates $x_i$, $y_i$  and $z_i$  due to the $j$th single nucleus at the coordinates $(x^{\prime}_{j}, y^{\prime}_{j}, z^{\prime}_{j})$ can be written as:
\begin{equation}
{\bf{u}}_{i} (x_i,y_i,z_i) = {\bf{u_{1}}}_{i} (x_i,y_i,z_i) + {\bf{u_{2}}}_{i} (x_i,y_i,z_i), 
\label{eq:nucleus-ui_total}
\end{equation}
where ${\bf{u_{1}}}_{i} (x_i,y_i,z_i) \equiv {\bf{u_{1}}}_{i} $ is the gradient of the function $\phi_1$ (equation \ref{eq:nucleus-phi1}) 
\begin{equation}
{\bf{u_{1}}}_{i}  = {\bf{\nabla}} \: {\phi_{1}(x_{i}, y_{i}, z_{i}, x^{\prime}_{j}, y^{\prime}_{j}, z^{\prime}_{j})} \:
= \frac{A  \: (1 + \nu)}{E} \: {\bf{\nabla}} \bigg( {\frac{1}{{R_1}_{ij}}} \bigg) 
\: \Delta p_{j} \: \:dv_j
\label{eq:nucleus-u1}
\end{equation}
and ${\bf{u_{2}}}_{i} (x_i,y_i,z_i) \equiv {\bf{u_{2}}}_{i} $ is obtained by  applying the operator ${\bf{{\nabla}_{2}}}$ (equation \ref{eq:nabla2}) to the imagem potential $\phi_2$ (equation \ref{eq:nucleus-phi2})
\begin{equation}
{\bf{u_{2}}}_{i} = {\bf{{\nabla}_{2}}} \: {\phi_{2}(x_{i}, y_{i}, z_{i}, x^{\prime}_{j}, y^{\prime}_{j}, z^{\prime}_{j})} \:  
= \frac{A  \: (1 + \nu)}{E} \: 
\bigg[ 
(3 - 4\nu){\bf{\nabla}} \bigg(  {\frac{1}{{R_2}_{ij}}} \bigg)   \:  
+ 2 {\bf{\nabla}} \bigg( z \frac{\partial }{\partial z} {\frac{1}{{R_2}_{ij}}} \bigg) \: 
- 4(1- \nu){\bf{\hat{z}}} {\bf{{\nabla}^{2}}}  \bigg( {\frac{z}{{R_2}_{ij}}} \bigg)
\bigg]   \: \Delta p_{j} \: \:dv_j , 
\label{eq:nucleus-u2}
\end{equation}
where  $A$ is a constant given by: 
\begin{equation}
A = - \frac{C_m E}{4 \pi (1 + \nu)}.  
\label{eq:nucleus-A}
\end{equation}
The elements  of the displacement vector ${\bf{u_{1}}}_{i}$ (equation \ref{eq:nucleus-u1}) are
\begin{equation}
{\bf{u_{1}}}_{i}  = 
\begin{bmatrix} 
{u_{1}}_x \\
{u_{1}}_y \\
{u_{1}}_z
\end{bmatrix}
 = \frac{A  \: (1 + \nu)}{E} \: 
\begin{bmatrix} 
\frac{\partial }{\partial x} {\frac{1}{{R_1}_{ij}}}  \\
\frac{\partial }{\partial y} {\frac{1}{{R_1}_{ij}}}  \\
\frac{\partial }{\partial z} {\frac{1}{{R_1}_{ij}}} 
\end{bmatrix}
 \: \Delta p_{j} \: \:dv_j \:  ,  
\label{eq:nucleus-u1-vector}
\end{equation}
where ${u_{1}}_x$, ${u_{1}}_y$ and ${u_{1}}_z$ are the $x-$, $y-$  and $z-$ components
of  ${\bf{u_{1}}}_{i} $ that gives the displacement field at the coordinates $x_i$, $y_i$  and $z_i$  due to the $j$th single nucleus in the infinite space.

The elements  of the displacement vector ${\bf{u_{2}}}_{i} $ (equation \ref{eq:nucleus-u2}) are
\begin{equation}
{\bf{u_{2}}}_{i}  = 
\begin{bmatrix} 
{u_{2}}_x \\
{u_{2}}_y \\
{u_{2}}_z
\end{bmatrix}
 = \frac{A  \: (1 + \nu)}{E} \: \left\{   (3  - 4 \nu)
\begin{bmatrix} 
  \: \frac{\partial }{\partial x} {\frac{1}{{R_2}_{ij}}}  \\
  \: \frac{\partial }{\partial y} {\frac{1}{{R_2}_{ij}}}  \\
 - \frac{\partial }{\partial z} {\frac{1}{{R_2}_{ij}}}  
\end{bmatrix}
+ 2 \: z_{i}
\begin{bmatrix} 
\frac{\partial^{2}  }{\partial x \partial z} {\frac{1}{{R_2}_{ij}}} \\
\frac{\partial^{2} }{\partial y \partial z} {\frac{1}{{R_2}_{ij}}} \\
\frac{\partial^{2} }{\partial z^{2}} {\frac{1}{{R_2}_{ij}}} 
\end{bmatrix}
\right\}
\: \Delta p_{j} \: \:dv_j,
\label{eq:nucleus-u2-vector}
\end{equation}
where ${u_{2}}_x$, ${u_{2}}_y$ and ${u_{2}}_z$ are the $x-$, $y-$  and $z-$ components
of  ${\bf{u_{2}}}_{i} $ that gives the correction of the displacements considering  a semi-space (image nucleus solution).

By following \cite{Sharma56} and \cite{Tempone10}, the Beltrami’s equations \citep{Beltrami} and the equilibrium equations must be satisfied to obtain the contribution of the stress field in the half space. 
The stress field  at the coordinates $x_i$, $y_i$ and $z_i$  due to the $j$th single nucleus of strain buried in the half space is given by
\begin{equation}
\mbox{\boldmath$\sigma$}_{i}(x_i,y_i,z_i) = \mbox{\boldmath$\sigma_{1}$}(x_i,y_i,z_i) + 
\mbox{\boldmath$\sigma_{2}$}(x_i,y_i,z_i) 
\label{eq:stress_nucleus}
\end{equation}
where $\mbox{\boldmath$\sigma_{1}$}(x_i,y_i,z_i) \equiv \mbox{\boldmath$\sigma_{1}$}_{i}$ represents the stress in an infinite medium, and 
$\mbox{\boldmath$\sigma_{2}$}(x_i,y_i,z_i) \equiv \mbox{\boldmath$\sigma_{2}$}_{i}$ represents a correction of the stress in a half-space due to an image nucleus.
Besides the Beltrami’s equations \citep{Beltrami} and the equilibrium equations that must be satisfied, the following boundary conditions at the free surface ($z_i = 0$) must be satisfied, i.e.:
\begin{equation}
\mbox{\boldmath$\sigma$}_{i}(x_i,y_i,0) = \mbox{\boldmath$\sigma_{1}$}(x_i,y_i,0) + 
\mbox{\boldmath$\sigma_{2}$}(x_i,y_i,0) \: = \: \mbox{ \boldmath$0$},
\label{eq:null_stress}
\end{equation}
where $\mbox{ \boldmath$0$}$ is the null vector that represents the null stress at the coordinates $x_i$, $y_i$ and $z_i = 0$.

The elements  of the stress vector $\mbox{\boldmath$\sigma_{1}$}_{i}$ (equation \ref{eq:stress_nucleus}) are
\begin{equation}
\mbox{\boldmath$\sigma_{1}$}_{i}  = 
\begin{bmatrix} 
{\widehat{xz}_{1}} \\
{\widehat{yz}_{1}}  \\
{\widehat{zz}_{1}} 
\end{bmatrix}
 = A  \: 
\begin{bmatrix} 
\frac{\partial^{2} }{\partial x \partial z} {\frac{1}{{R_1}_{ij}}}  \\
\frac{\partial^{2} }{\partial y \partial z} {\frac{1}{{R_1}_{ij}}}  \\
\frac{\partial^{2} }{\partial z^{2}} {\frac{1}{{R_1}_{ij}}} 
\end{bmatrix}
 \: \Delta p_{j} \: \:dv_j \:  ,  
\label{eq:nucleus-stress1-vector}
\end{equation}
where ${\widehat{xz}_{1}}$, ${\widehat{yz}_{1}}$ and ${\widehat{zz}_{1}}$ are the $x-$, $y-$  and $z-$components
of  $\mbox{\boldmath$\sigma_{1}$}_{i}$ that gives the stress in an infinite medium due to the $j$th nucleus of strain.

The elements  of the stress vector $\mbox{\boldmath$\sigma_{2}$}_{i}$ (equation \ref{eq:stress_nucleus}) are
\begin{equation}
\mbox{\boldmath$\sigma_{2}$}_{i}  = 
\begin{bmatrix} 
{\widehat{xz}_{2}} \\
{\widehat{yz}_{2}}  \\
{\widehat{zz}_{2}} 
\end{bmatrix}
= A  \: 
\left\{
\begin{bmatrix} 
\frac{\partial^{2}  }{\partial x \partial z} {\frac{1}{{R_2}_{ij}}} \\
\frac{\partial^{2} }{\partial y \partial z} {\frac{1}{{R_2}_{ij}}} \\
- \frac{\partial^{2} }{\partial z^{2}} {\frac{1}{{R_2}_{ij}}}   
\end{bmatrix}
+ 2 \: z_{i}
\begin{bmatrix} 
\frac{\partial^{3}  }{\partial x \partial z^{2}} {\frac{1}{{R_2}_{ij}}} \\
\frac{\partial^{3} }{\partial y \partial z^{2}} {\frac{1}{{R_2}_{ij}}} \\
\frac{\partial^{3} }{\partial z^{3}} {\frac{1}{{R_2}_{ij}}} 
\end{bmatrix}
\right\}
\: \Delta p_{j} \: \:dv_j,
\label{eq:nucleus-stress2-vector}
\end{equation}
where ${\widehat{xz}_{2}}$, ${\widehat{yz}_{2}}$ and ${\widehat{zz}_{2}}$ are the $x-$, $y-$  and $z-$components
of  $\mbox{\boldmath$\sigma_{2}$}_{i}$ that gives the correction of the stress considering  a semi-space due to the $j$th image nucleus.

We validate our equations by verifying if the null stress acting through the free surface (equation \ref{eq:null_stress}) is satisfied.
Hence, by taking the elements of stress vectors $\mbox{\boldmath$\sigma_{1}$}_{i}$ and 
$\mbox{\boldmath$\sigma_{2}$}_{i}$  at the $i$th coordinates of the free surface ($x_i$, $y_i$ and $z_i = 0$) the following relationship:
\begin{equation}
{\widehat{xz}_{1}} + {\widehat{xz}_{2}} = {\widehat{yz}_{1}} + {\widehat{yz}_{2}}
+ {\widehat{zz}_{1}} + {\widehat{zz}_{2}}  = 0
\label{eq:null_stress_i}
\end{equation}
must be met.


\subsection{The discrete displacement forward modeling due to a reservoir in a homogeneous elastic semi-infinite medium} \label{u-model}


We parametrized the reservoir as a grid of juxtaposed right rectangular prisms.
Each grid prism has homogeneous pressure contrasts; however, pressure variations among different prisms are allowed. 
To calculate the displacement due to the pressure change in the whole reservoir with this discretization model, we use the solution deduced for a single nucleus of strain in a homogeneous elastic semi-infinite medium (subsection \ref{solution-nucleus}) in the following way. 
First, we assume that the coordinates of the $j$th prism center are the coordinates of a nucleus of strain.
Next, the displacement field calculated at the $i$th coordinates  $(x_i$, $y_i, z_i)$  due to the pressure contrast of the $j$th prism is calculated with a integration over its volume.
Then, from equation \ref{eq:nucleus-ui_total}, the displacement field produced by the $jth$ prism can be written as 
\begin{equation}
{\bf{u}}_{i} (x_i,y_i,z_i) =  
\int_{zo_j - \Delta z/2}^{zo_j + \Delta z/2} \:\:
\int_{yo_j - \Delta y/2}^{yo_j + \Delta y/2} \:\: 
\int_{xo_j - \Delta x/2}^{xo_j + \Delta x/2} 
{\bf{u_{1}}}_{i} \: \:  dx^{\prime}_j dy^{\prime}_j dz^{\prime}_j 
\: + \:
\int_{- zo_j - \Delta z/2}^{- zo_j + \Delta z/2} \:\:
\int_{yo_j - \Delta y/2}^{yo_j + \Delta y/2} \:\: 
\int_{xo_j - \Delta x/2}^{xo_j + \Delta x/2} 
{\bf{u_{2}}}_{i} \:\:  dx^{\prime}_j dy^{\prime}_j dz^{\prime}_j.
\label{eq:nucleus-ui_jth_prism}
\end{equation}

In equation \ref{eq:nucleus-ui_jth_prism},  ${\bf{u_{1}}}_{i}$ is the displacement field at the coordinates $x_i$, $y_i$  and $z_i$  due to the $j$th single nucleus in the infinite space (equations \ref{eq:nucleus-u1} and  \ref{eq:nucleus-u1-vector}),  $xo_j, yo_j$ and $zo_j$ are, respectively, the $x-, y-$ and $z-$coordinates of the $j$th prism center and $\Delta x, \: \Delta y$ and $\Delta z$ are the dimensions of the prisms along the $x-, y-$ and $z-$directions, respectively. 
Additionally, ${\bf{u_{2}}}_{i}$ is the displacement field at the coordinates $x_i$, $y_i$  and $z_i$  due to the effect of an image nucleus (equations \ref{eq:nucleus-u2} and  \ref{eq:nucleus-u2-vector}), where $xo_j, yo_j$ and $- zo_j$ are, respectively, the $x-, y-$ and $z-$coordinates of the $j$th image nucleus. 
 
Note that the integrations in equation \ref{eq:nucleus-ui_jth_prism}, are conducted with respect to the variables $(x^{\prime}_{j}, y^{\prime}_{j}, z^{\prime}_{j})$, denoting, respectively, the
$x-, \: y-$, and $z-$coordinates of an arbitrary point belonging to the interior
of the $j$th prism (or the $j$th image nucleus).
Finally, the displacement field at the coordinates $x_i$, $y_i$  and $z_i$ due to the pressure change in the whole reservoir can be defined as the sum of the displacements yielded by each prism with constant pressure:
\begin{equation}
{\bf{\tilde{u}}}_{i} (x_i,y_i,z_i) = \sum_{j=1}^{M} 
\int_{zo_j - \Delta z/2}^{zo_j + \Delta z/2} \:\:
\int_{yo_j - \Delta y/2}^{yo_j + \Delta y/2} \:\: 
\int_{xo_j - \Delta x/2}^{xo_j + \Delta x/2} 
{\bf{u_{1}}}_{i} \: \:  dx^{\prime}_j dy^{\prime}_j dz^{\prime}_j 
\: + \: 
\int_{- zo_j - \Delta z/2}^{- zo_j + \Delta z/2} \:\:
\int_{yo_j - \Delta y/2}^{yo_j + \Delta y/2} \:\: 
\int_{xo_j - \Delta x/2}^{xo_j + \Delta x/2} 
{\bf{u_{2}}}_{i} \:\:  dx^{\prime}_j dy^{\prime}_j dz^{\prime}_j, 
\label{eq:nucleus-ui_M_prisms}
\end{equation}
where $M$ is the number of prisms setting up the reservoir model.

From equation \ref{eq:nucleus-u1-vector},  we can get the   
$\alpha-$components of the displacement vectors ${\bf{u_{1}}}_{i}$  (the integrand of the first integral of equation \ref{eq:nucleus-ui_M_prisms})
\begin{equation}
{u_{1}}_\alpha =  \frac{A  \: (1 + \nu)}{E} \: \Bigg[  \frac{\partial }{\partial \alpha} {\frac{1}{{R_1}_{ij}}} \Bigg] \: \Delta p_{j} \:\:\:  
\label{eq:u1_alpha_component}
\end{equation}
where $\alpha$  belongs to the set of $ x-$ $y-$ and $z-$directions of the Cartesian coordinates system.

From equation  \ref{eq:nucleus-u2-vector}, we can get the $ x-$ and $y-$components of the displacement vectors ${\bf{u_{2}}}_{i}$  (the integrand of the second integral of equation \ref{eq:nucleus-ui_M_prisms})
\begin{equation}
{u_{2}}_\alpha =  \frac{A  \: (1 + \nu)}{E} \: \Bigg[ 
(3  - 4 \nu)    \: \frac{\partial }{\partial \alpha} {\frac{1}{{R_2}_{ij}}}\: 
+ 2 \: z_{i} \:  
\frac{\partial^{2}  }{\partial \alpha \partial z} {\frac{1}{{R_2}_{ij}}}  \: \Bigg]
\Delta p_{j} ,
\label{eq:u2_alpha_component}
\end{equation}
where  $\alpha$  belongs to the set of $ x-$ and $y-$directions, and the $z-$component of the displacement vectors ${\bf{u_{2}}}_{i}$  
\begin{equation}
{u_{2}}_z =  \frac{A  \: (1 + \nu)}{E} \: \Bigg[ 
(3  - 4 \nu)    \: - \frac{\partial }{\partial z} {\frac{1}{{R_2}_{ij}}}\: 
+ 2 \: z_{i} \:  
\frac{\partial^{2}  }{\partial z^{2}} {\frac{1}{{R_2}_{ij}}}  \: \Bigg]
\Delta p_{j}
\label{eq:u2_z_component}
\end{equation}


Equation \ref{eq:u1_alpha_component} shows that the $\alpha-$component of the displacement vector ${\bf{u_{1}}}_{i}$  depends on the first derivative of ${\frac{1}{{R_1}_{ij}}}$  
with respect to the variable $\alpha$.
Conversely, the $\alpha-$component of the displacement vector ${\bf{u_{2}}}_{i}$  
(equation \ref{eq:u2_alpha_component}) depends not only on the first derivative of 
${\frac{1}{{R_2}_{ij}}}$  with respect to the variable $\alpha$ but also on the second derivative of ${\frac{1}{{R_2}_{ij}}}$  with respect to the variables $\alpha$ and $z$.

By substituting equations \ref{eq:u1_alpha_component} $-$ \ref{eq:u2_z_component} into equation \ref{eq:nucleus-ui_M_prisms}, we can obtain, respectively,  the $\alpha-$component (where  $\alpha = x$ and $y$) and the $z-$ component of the displacement field at the $i$th coordinates ($x_i$, $y_i$  and $z_i$) due to the pressure change in the whole reservoir, i.e.:
\begin{equation}
{\tilde{u}}_{{i}_\alpha} = \frac{A  \: (1 + \nu)}{E}  \sum_{j=1}^{M} 
\: \Delta p_{j}  \: \int\int\limits_{v_j}\int 
\frac{\partial }{\partial \alpha} {\frac{1}{{R_1}_{ij}}} \:  dv_j 
\: + \: 
(3  - 4 \nu)  \int\int\limits_{v_j}\int
\: \frac{\partial }{\partial \alpha} {\frac{1}{{R_2}_{ij}}}\:  dv_j
+ \: 2 \: z_{i} \:  \int\int\limits_{v_j}\int
\frac{\partial^{2}  }{\partial \alpha \partial z} {\frac{1}{{R_2}_{ij}}}  \:\:  dv_j, 
\label{eq:u_til_alpha}
\end{equation}
and 
\begin{equation}
{\tilde{u}}_{{i}_z} = \frac{A  \: (1 + \nu)}{E}  \sum_{j=1}^{M} 
\: \Delta p_{j}  \: \int\int\limits_{v_j}\int 
\frac{\partial }{\partial z} {\frac{1}{{R_1}_{ij}}} \:  dv_j 
\: - \: (3  - 4 \nu)  \int\int\limits_{v_j}\int
\: \frac{\partial }{\partial z} {\frac{1}{{R_2}_{ij}}}\:  dv_j
+ \: 2 \: z_{i} \:  \int\int\limits_{v_j}\int
\frac{\partial^{2}  }{\partial z^{2}} {\frac{1}{{R_2}_{ij}}}  \:\:  dv_j, 
\label{eq:u_til_z}
\end{equation}
where $dv_j$ is the $j$th element of volume of the $j$th prism whose volume is $v_j$.


In the right-hand side of equations \ref{eq:u_til_alpha} and \ref{eq:u_til_z}, the three integrals are equal to  quantities of the gravitational attraction produced by the $j$th prism considering that 
$\Delta p_{j}$ is the density of the $j$th prism and the constants are equivalent to the gravitational constant. 
The first integral corresponds to the $\alpha-$component of the gravitational attraction produced by the $j$th prism.
The second and third integrals correspond, respectively, to the $\alpha-$component of the gravitational attraction and to the $\alpha \:z-$component of the gravity gradient tensor produced by the $j$th image nucleus. 

The similarity between the displacement fields due to a volume source in a half-space and the gravity field allows the use of closed expressions of the gravitational potential and its derivatives produced by the 3D right rectangular prism derived by \cite{Nagyetal2000} and (\citeyear{Nagyetal2002}).  

The first integral in the right-hand side of equations  \ref{eq:u_til_alpha} and \ref{eq:u_til_z}  is the first derivatives of ${\frac{1}{{R_1}_{ij}}}$ with respect to $x$, $y$ and $z$ are, respectively, given by the following closed expressions: 
\begin{equation}
\int\int\limits_{v_j}\int  \frac{\partial }{\partial x} {\frac{1}{{R_1}_{ij}}} dv_j =
\Bigg|\Bigg|\Bigg| 
y \ln(z + {R_1}) + z \ln(y + {R_1}) -  x  \tan^{-1} \Bigg( \frac{yz}{x \:{R_1}} \Bigg) 
\Bigg|_{x_1}^{x_2} \Bigg|_{y_1}^{y_2} \Bigg|_{z_1}^{z_2}
\label{dx1}
\end{equation}
\begin{equation}
\int\int\limits_{v_j}\int  \frac{\partial }{\partial y} {\frac{1}{{R_1}_{ij}}} dv_j =
\Bigg|\Bigg|\Bigg|
x \ln(z + {R_1}) + z \ln(x + {R_1}) -  y  \tan^{-1} \Bigg( \frac{xz}{y \:{R_1}} 
\Bigg)
\Bigg|_{x_1}^{x_2} \Bigg|_{y_1}^{y_2} \Bigg|_{z_1}^{z_2}
\label{dy1}
\end{equation}
\begin{equation}
\int\int\limits_{v_j}\int  \frac{\partial }{\partial z} {\frac{1}{{R_1}_{ij}}} dv_j =
\Bigg|\Bigg|\Bigg|
x \ln(y + {R_1}) + y \ln(x + {R_1}) -  z  \tan^{-1} \Bigg( \frac{xy}{z \:{R_1}} \Bigg) 
\Bigg|_{x_1}^{x_2} \Bigg|_{y_1}^{y_2} \Bigg|_{z_1}^{z_2}
\label{dz1}
\end{equation}
For simplicity,  we omit the subscripts $i$ and $j$ in equations \ref{dx1} $-$ \ref{dz1}; hence, the variables $x$, $y$ and $z$  are relative coordinates of the $i$th point of the displacement referred to the coordinates to the corner of the $j$th prism modeling the reservoir, i.e.,  $x = x_j - x_i $, $\: y = y_j - y_i $, $\: z = z_j - z_i $ and 
$\: {R_1} = \sqrt{x^{2} + y^{2} + z^{2}}$.
In equations \ref{dx1} $-$ \ref{dz1}, the limits of the integrals represent the borders of the $j$th prism modeling the reservoir in the following way: $x_1$ and $x_2$ are their south and north borders; $y_1$ and  $y_2$ are their  west and east borders;  and 
$z_1$ and $z_2$ are their depths to the top and bottom.
\cite{Nagyetal2000} and (\citeyear{Nagyetal2002}) provided limit values of integrals shown in equations \ref{dx1} $-$ \ref{dz1} when the computation point coincides with the corner of the prism.

The second and third integrals in the right-hand side of equations  \ref{eq:u_til_alpha}  and \ref{eq:u_til_z} are related with the correction of the displacements considering  a semi-space (image nucleus solution).
These integrals depend on the distance ${R_2}_{ij}$ from the $i$th coordinate point of the displacement to the $j$th image nucleus.
Likewise, $\: {R_2} = \sqrt{x^{2} + y^{2} + z^{2}}$; however, the variable $z$  that represents the relative coordinate of the $i$th point of the displacement referred to the coordinates to the corner of the $j$th image nucleus is given by
\begin{equation}
z = z_j - z_i - 2 z_c
\label{z_image}
\end{equation}
where  $z_c$ is the $z-$coordinate of the $j$th image nucleus
\begin{equation}
z_c = 0.5 (z_1 + z_2)
\label{z_c}
\end{equation}

The second integral in the right-hand side of  equations  \ref{eq:u_til_alpha} and  \ref{eq:u_til_z} is the first derivatives of ${\frac{1}{{R_2}_{ij}}}$ with respect to $x$, $y$ and $z$.
These derivatives are equal to equations \ref{dx1} $-$ \ref{dz1}, the only difference is the variable $z$  that is given by equations \ref{z_image}  and \ref{z_c}.

The third integral in the right-hand side of  equations  \ref{eq:u_til_alpha}  and  \ref{eq:u_til_z} is the second derivatives of ${\frac{1}{{R_2}_{ij}}}$ with respect to $xz$, $yz$ and $zz$ which are, respectively, given by the following closed expressions: 
\begin{equation}
\int\int\limits_{v_j}\int 
\frac{\partial^{2}  }{\partial x \partial z} {\frac{1}{{R_2}_{ij}}}  dv_j = \Bigg|\Bigg|\Bigg| 
\ln(y + {R_2})
\Bigg|_{x_1}^{x_2} \Bigg|_{y_1}^{y_2} \Bigg|_{z_1}^{z_2}
\label{dx2}
\end{equation}
\begin{equation}
\int\int\limits_{v_j}\int
\frac{\partial^{2}  }{\partial y \partial z} {\frac{1}{{R_2}_{ij}}}  dv_j =
\Bigg|\Bigg|\Bigg|
\ln(x + {R_2})
\Bigg|_{x_1}^{x_2} \Bigg|_{y_1}^{y_2} \Bigg|_{z_1}^{z_2}
\label{dy2}
\end{equation}
\begin{equation}
\int\int\limits_{v_j}\int
\frac{\partial^{2}  }{\partial z \partial z} {\frac{1}{{R_2}_{ij}}}  dv_j =
\Bigg|\Bigg|\Bigg|
 -   \tan^{-1} \Bigg( \frac{xy}{z \:{R_2}} \Bigg)
\Bigg|_{x_1}^{x_2} \Bigg|_{y_1}^{y_2} \Bigg|_{z_1}^{z_2}
\label{dz2}
\end{equation}

The horizontal displacement field at the $i$th coordinates ($x_i$, $y_i$  and $z_i$) due to the pressure change in the whole reservoir is calculated by 
\begin{equation}
{\tilde{u}}_{{i}_h} = \sqrt{ {\tilde{u}}_{{i}_x}^{2}  +  {\tilde{u}}_{{i}_y}^{2} }  
\label{eq:horizontal_displacement}
\end{equation}
where ${\tilde{u}}_{{i}_x}$ and ${\tilde{u}}_{{i}_y}$ are the $x-$ and $y-$ components of the displacement field at the $i$th coordinates given by equation \ref{eq:u_til_alpha}, with $\alpha = x $ and $y$.


\subsection{The discrete stress forward modeling due to a reservoir in a homogeneous elastic semi-infinite medium}

By following the similar approach used in the displacement forward modeling due to a prism in a homogeneous elastic semi-infinite medium (subsection \ref{u-model}), the stress field of each prism assuming constant pressure is calculated by integrating over its volume the stress of a nucleus of strain located at its center. 
Next, the stresses due to a set of $M$ prisms modeling the reservoir is summed to yield the stress field  due to the pressure change in the whole reservoir.

Like equation \ref{eq:nucleus-ui_M_prisms}, the stress field at the $i$th coordinates ($x_i$, $y_i$  and $z_i$) due to the pressure change in the whole reservoir can be written as:
\begin{equation}
\tilde{\mbox{\boldmath$\sigma$}}_{i}(x_i,y_i,z_i)
= \sum_{j=1}^{M} 
\int_{zo_j - \Delta z/2}^{zo_j + \Delta z/2} \:\:
\int_{yo_j - \Delta y/2}^{yo_j + \Delta y/2} \:\: 
\int_{xo_j - \Delta x/2}^{xo_j + \Delta x/2} 
\mbox{\boldmath$\sigma_{1}$}_{i}\: \:  dx^{\prime}_j dy^{\prime}_j dz^{\prime}_j 
\: + \: 
\int_{- zo_j - \Delta z/2}^{- zo_j + \Delta z/2} \:\:
\int_{yo_j - \Delta y/2}^{yo_j + \Delta y/2} \:\: 
\int_{xo_j - \Delta x/2}^{xo_j + \Delta x/2} 
\mbox{\boldmath$\sigma_{2}$}_{i} \:\:  dx^{\prime}_j dy^{\prime}_j dz^{\prime}_j, 
\label{eq:nucleus-stress_M_prisms}
\end{equation}

From equations \ref{eq:nucleus-stress1-vector} and \ref{eq:nucleus-stress2-vector}, we write, respectively,  the $\alpha-$component (where  $\alpha = x$ and $y$) and the $z-$ component of the stress field at the $i$th coordinates ($x_i$, $y_i$  and $z_i$) due to the pressure change in the whole reservoir, i.e.:
\begin{equation}
{\tilde{\sigma}}_{{i}_\alpha} = A \: \sum_{j=1}^{M} 
\: \Delta p_{j}  \: \int\int\limits_{v_j}\int 
\frac{\partial^{2}}{\partial \alpha \partial z} {\frac{1}{{R_1}_{ij}}} \:  dv_j 
\: +   \int\int\limits_{v_j}\int
\: \frac{\partial^{2} }{\partial \alpha \partial z} {\frac{1}{{R_2}_{ij}}}\:  dv_j
+ \: 2 \: z_{i} \:  \int\int\limits_{v_j}\int
\frac{\partial^{3}  }{\partial \alpha \partial z ^{2}} {\frac{1}{{R_2}_{ij}}}  \:\:  dv_j, 
\label{eq:stress_til_alpha}
\end{equation}
and 
\begin{equation}
{\tilde{\sigma}}_{{i}_z} = A \: \sum_{j=1}^{M} 
\: \Delta p_{j}  \: \int\int\limits_{v_j}\int 
\frac{\partial^{2}}{\partial z^{2}} {\frac{1}{{R_1}_{ij}}} \:  dv_j 
\: -   \int\int\limits_{v_j}\int
\: \frac{\partial^{2} }{\partial z^{2}} {\frac{1}{{R_2}_{ij}}}\:  dv_j
+ \: 2 \: z_{i} \:  \int\int\limits_{v_j}\int
\frac{\partial^{3}  }{\partial z ^{3}} {\frac{1}{{R_2}_{ij}}}  \:\:  dv_j, 
\label{eq:stress_til_z}
\end{equation}
where $dv_j$ is the $j$th element of volume of the $j$th prism whose volume is $v_j$.

Like in the displacement field (equations \ref{eq:u_til_alpha} and \ref{eq:u_til_z}), the three integrals, in the right-hand side of equations \ref{eq:stress_til_alpha} and \ref{eq:stress_til_z}, are equal to  quantities of the gravitational attraction produced by the $j$th prism considering that $\Delta p_{j}$ is the density of the $j$th prism and the constants are equivalent to the gravitational constant. 
In equation \ref{eq:stress_til_alpha}, the first and second integrals correspond to the $\alpha \:z-$component of the gravity gradient tensor (where  $\alpha = x$ and $y$) produced, respectively, by the $j$th prism and $j$th image nucleus; and the third integral corresponds to the first derivative with respect to $\alpha$ of the $zz-$component of the gravity gradient tensor produced by the $j$th image nucleus. 
In equation \ref{eq:stress_til_z}, the first and second integrals correspond to the $zz-$component of the gravity gradient tensor produced, respectively, by the $j$th prism and $j$th image nucleus; and the third integral corresponds to the first derivative with respect to $z$ of the $zz-$component of the gravity gradient tensor produced by the $j$th image nucleus. 

Here, we also use the closed expressions of the gravitational potential and its derivatives produced by the 3D right rectangular prism derived by \cite{Nagyetal2000} and (\citeyear{Nagyetal2002}) to calculate the stress field due to a volume source in a half-space.  
The first integral in equations  \ref{eq:stress_til_alpha} and \ref{eq:stress_til_z}
corresponds, respectively, to the $\alpha \:z-$ and $zz-$components of the gravity gradient tensor produced by the $j$th prism modeling the reservoir.
These second partial derivatives of ${\frac{1}{{R_1}_{ij}}}$ with respect to $xz$, $yz$ and $zz$ are calculated using the analytical expressions given by equations \ref{dx2} $-$ \ref{dz2} but substituting 
$\frac{1}{{R_2}_{ij}} \equiv \frac{1}{R_2}$ by 
$\frac{1}{{R_1}_{ij}} \equiv \frac{1}{R_1}$.

The second and third integrals in the right-hand side of  equations  \ref{eq:stress_til_alpha}  and \ref{eq:stress_til_z} are related with the correction of the stresses considering  a semi-space (image nucleus solution).
These integrals depend on the distance ${R_2}_{ij}$ from the $i$th coordinate point of the stress to the $j$th image nucleus, where the variable $z$ is given by the equations \ref{z_image} and \ref{z_c}.
In the right-hand side of  equations  \ref{eq:stress_til_alpha} and \ref{eq:stress_til_z}, 
the second integral is the second partial derivatives of ${\frac{1}{{R_2}_{ij}}}$ with respect to $xz$, $yz$ and $zz$ given by equations  \ref{dx2} $-$ \ref{dz2} and the third integral is the third partial derivatives of ${\frac{1}{{R_2}_{ij}}}$ with respect to $xzz$, $yzz$ and $zzz$ given by (\cite{Nagyetal2000}):
\begin{equation}
\int\int\limits_{v_j}\int 
\frac{\partial^{3}}{\partial x \partial z^{2}} {\frac{1}{{R_2}_{ij}}}  dv_j =
\Bigg|\Bigg|\Bigg| 
\frac{- y z}{R_2} \Big( \frac{1}{x^{2} + z^{2}} \Big)
\Bigg|_{x_1}^{x_2} \Bigg|_{y_1}^{y_2} \Bigg|_{z_1}^{z_2}
\label{sx3}
\end{equation}
\begin{equation}
\int\int\limits_{v_j}\int  
\frac{\partial^{2}  }{\partial y \partial z^{2}} {\frac{1}{{R_2}_{ij}}}  dv_j =
\Bigg|\Bigg|\Bigg|
\frac{- x z}{R_2} \Big( \frac{1}{y^{2} + z^{2}} \Big)
\Bigg|_{x_1}^{x_2} \Bigg|_{y_1}^{y_2} \Bigg|_{z_1}^{z_2}
\label{sy3}
\end{equation}
\begin{equation}
\int\int\limits_{v_j}\int 
\frac{\partial^{3}  }{\partial z^{3}} {\frac{1}{{R_2}_{ij}}}  dv_j =
\Bigg|\Bigg|\Bigg|
\frac{ x y}{R_2} \Big( \frac{1}{x^{2} + z^{2}} + \frac{1}{y^{2} + z^{2}} \Big)
\Bigg|_{x_1}^{x_2} \Bigg|_{y_1}^{y_2} \Bigg|_{z_1}^{z_2}
\label{sz3}
\end{equation}

\subsection{Computation notes}

In equations \ref{dx1}$-$\ref{dz1} and \ref{dx2}$-$\ref{dz1}, we adopted the modifications proposed by \cite{Fukushima2020}.
To overcome the zero division in evaluating the arguments of the arctangent function, \cite{Fukushima2020} replaced  $\tan^{-1} \big( \frac{S}{T} \big)$ by 
\begin{equation}
arctan2(S,T) = 
  \left\{ \begin{aligned}
      atan (S/T) & \:\: \mbox{if} \: \:\:  T \neq 0 \\
      \pi /2 & \:\: \mbox{if} \: \:\:  T = 0  \: \: \mbox{and} \: \:S > 0 \\
      -\pi /2 & \:\: \mbox{if} \: \:\:  T = 0  \: \: \mbox{and} \: \:S < 0 \\
         0 & \:\: \mbox{if} \: \:\:  T = 0  \: \: \mbox{and} \: \:S = 0 \\
  \end{aligned} \right.
\label{eq:arctan2}  
\end{equation}
If the argument of the logarithm is less than $10^{-10}$, the logarithm is replaced by zero; otherwise the logarithm is  calculated regularly



\section{NUMERICAL APPLICATIONS}

\subsubsection{Disk-shaped reservoir under uniform depletion}

Embedded in a semi-infinite homogenous medium, we simulated a vertical cylinder-like reservoir (Figure \ref{fig:cylinder}) with a radius of 500 m and whose horizontal coordinates of its center along the north-south and east-west directions are 0 m and 0 m, respectively.
The depths to the top and to the bottom of the simulated reservoir are 750 m and 850 m, respectively.
The reservoir is uniformly depleted by $\Delta p = -10$ MPa. 
The Young’s modulus is  3300 (in MPa), the Poisson's coefficient is 0.25,
the uniaxial compaction coefficient $C_{m}$  (equation \ref{eq:Cm}) is $2.2525 \: 10^{-4}$
$\textrm{ MPa}^{-1}$.

\begin{figure}[h]
\includegraphics[width=8.3cm]{Fig/Figure_Cylinder.png}
\caption{Disk-shaped reservoir under uniform depletion with a radius of 500 m}
\label{fig:cylinder}
\end{figure}


We  discretized the cylinder  along the $x-$ and $y-$ directions into an $20 \times 20$ grid of prisms. 
Hence, we totalized 400 prisms all of them centered at 800 m deep, with depths to the top and to the bottom at 750 m and 850 m  and with pressure contrast 
$\Delta p_j$, $j = 1, ..., 400$ equal to $-10$ MPa.
 
We calculate the displacement fields due to the pressure change in the whole cylindrical reservoir.
Figures \ref{fig:displacement}  and \ref{fig:displacement_Geertsma}  show cross-sections at 
$x  = 0$ m of the displacement fields in 2D contour plots  due to the pressure change in the whole reservoir by using our methodology and Geertsma’s method \citep{Geertsma73}, respectively.
To use the Geertsma’s method \citep{Geertsma73}, we used a regular grid of $20 \times 20$ nuclei along the $x-$ and $y-$ directions all of them centered at 800 m deep. 
Because we defined the $z-$axis as positive downwards, the positive vertical displacement means a subsidence and the negative vertical displacement means an uplift.
Figure \ref{fig:displacement}  shows the horizontal and vertical displacements  calculated, respectively, with equations \ref{eq:horizontal_displacement} and \ref{eq:u_til_z} by our methodology that uses the closed expressions of the full integrations (equations \ref{eq:u_til_alpha} and \ref{eq:u_til_z}) of \cite{Nagyetal2000} and \cite{Nagyetal2002} (equations \ref{dx1} $-$ \ref{dz2}).

Figure \ref{fig:displacement_Geertsma} shows the radial and vertical displacements using Geertsma’s method \citep{Geertsma73}  considering an elastic homogeneous cylindrical reservoir under uniform depletion based on the nucleus-of-strain concept in the half-space, which in turn was proposed by \cite{Mindlin&Cheng50} and \cite{Sen51} in the theory of thermoelasticity.

In both cases (Figures \ref{fig:displacement}b and \ref{fig:displacement_Geertsma}b) the vertical displacements due to the entire the disc-shaped reservoir display a subsidence (positive values) above the reservoir and an uplift (negative values) below the reservoir.
We stress that the proposed volume integrations  (equations \ref{dx1} $-$ \ref{dz2})  allowed  to evaluate the  vertical displacement (Figure \ref{fig:displacement}b) throughout  the entire reservoir including inside and outside the reservoir.
Rather, the  vertical displacement using Geertsma’s method 
(Figure \ref{fig:displacement_Geertsma}b) is only valid outside the reservoir. 

The radial displacement using Geertsma’s method 
(Figure \ref{fig:displacement_Geertsma}a) shows positive values at the edges of the reservoir ($y= -500$ and $y = 500$) with a singularity at the center of the reservoir 
($x= 0, \: y = 0$ and $z = 800$ m). 
The horizontal displacement with the proposed full integration 
(Figures \ref{fig:displacement}a) shows positive values at the edges of the reservoir ($y= -500$ and $y = 500$); however, it does not present sigularities inside the reservoir.
 
Figure \ref{fig:displacement_z_levels} shows the $x-$component displacement and vertical displacement by our methodology that uses a full volume integrations.
These displacements are calculated along the $x-$axis, at $y = 0$ m and considering four surfaces located at the following depths:  seafloor ($z = 0$ m), reservoir top ($z = 750$ m), reservoir center ($z = 800$ m) and reservoir bottom ($z = 850$ m).
In the $x$-component of the displacement (Figure \ref{fig:displacement_z_levels}a), we can note an increased horizontal contraction from the center of the reservoir ($x = 0$) toward the reservoir edge ($x= 500$ m) where the maximum contraction of all surfaces occur.
In the vertical displacement (Figure \ref{fig:displacement_z_levels}b), we can note a subsidence of the seafloor and the reservoir top (positive values) and an uplift of the reservoir bottom (negative values).
The vertical displacements of the seafloor, the top and bottom of the reservoir for Geertsma’s method (Figure \ref{fig:displacement_z_levels_Geertsma}) show a similar behavior of those obtained by our methodology that uses a full volume integrations (Figure \ref{fig:displacement_z_levels}b). 
However, we note that the subsidence of the seafloor is more attenuated in the Geertsma’s method than in our method.
This fact is important because the moviment of the seafloor should be monitored in hydrocarbon fields under production.

Figure \ref{fig:Null_stress} shows the  null stress through the free surface at the plane 
$z=0$ m (equations \ref{eq:null_stress} and \ref{eq:null_stress_i}) due to reservoir under uniform depletion.

\begin{figure}[h]
\includegraphics[width=8.3cm]{Fig/Figure_Displacement.png}
\caption{Reservoir under uniform depletion: (a) Horizontal displacement (equation \ref{eq:horizontal_displacement}) and (b) vertical displacement (equation \ref{eq:u_til_z}) by our methodology that uses the closed expressions of the volume integrations given by \cite{Nagyetal2000} and \cite{Nagyetal2002} (equations \ref{dx1}-\ref{dz2})}
\label{fig:displacement}
\end{figure}

\begin{figure}[h]
\includegraphics[width=8.3cm]{Fig/Figure_Displacement_Geertsma.png}
\caption{Reservoir under uniform depletion: (a) Radial displacement and (b) vertical displacement using Geertsma’s method \citep{Geertsma73}  considering an elastic homogeneous cylindrical reservoir under uniform depletion based on the nucleus-of-strain concept in the half-space (\cite{Mindlin&Cheng50} and \cite{Sen51})}
\label{fig:displacement_Geertsma}
\end{figure}

\begin{figure}[h]
\includegraphics[width=8.3cm]{Fig/Figure_Displacement_z_levels.png}
\caption{Reservoir under uniform depletion: (a) Horizontal x-component displacement and (b) vertical displacement by our methodology that uses the closed expressions of the volume integrations given by \cite{Nagyetal2000} and \cite{Nagyetal2002} (equations \ref{dx1}-\ref{dz2}).
These displacements are calculated along the x-axis, at $y = 0$ m and $z$ located at the depths of:  seafloor ($z = 0$ m), reservoir top ($z = 750$ m), reservoir center ($z = 800$ m) and reservoir bottom ($z = 850$ m).}
\label{fig:displacement_z_levels}
\end{figure}
\begin{figure}[h]
\includegraphics[width=8.3cm]{Fig/Figure_Displacement_z_levels_Geertsma.png}
\caption{Reservoir under uniform depletion: Vertical displacement using Geertsma’s method \citep{Geertsma73}  considering an elastic homogeneous cylindrical reservoir under uniform depletion based on the nucleus-of-strain concept in the half-space (\cite{Mindlin&Cheng50}
The displacement is calculated along the x-axis, at $y = 0$ m and $z$ located at the depths of:  seafloor ($z = 0$ m), reservoir top ($z = 750$ m), and reservoir bottom 
($z = 850$ m).}
\label{fig:displacement_z_levels_Geertsma}
\end{figure}


\begin{figure*}[h]
\includegraphics[width=12cm]{Fig/Figure_Null_stress.png}
\caption{Reservoir under uniform depletion: (a) $x-$, (b) $y-$, and (c) $z-$ components of the stress at the free surface.}
\label{fig:Null_stress}
\end{figure*}

\subsubsection{Disk-shaped reservoir under non uniform depletion}

Here, we kept the same dimensions of the cylindrical reservoir simulated previously.
We also kept the  reservoir properties, except the pressure.
We simulated a non-uniform depletion scenario where the cylindrical reservoir is composed by two vertically juxtaposed cylinders, each one with a uniform depletion.
The deepest cylinder is uniformly depleted by $\Delta p = -20$ MPa with its top and bottom at, respectively, 800 and 850 m deep.
The shallowest cylinder is uniformly depleted by $\Delta p = -40$ MPa with its top and bottom at, respectively, 750 and 800 m deep.

We  discretized the cylinder  along the $x-$, $y-$ and $z-$ directions into an $20 \times 20 \times 2$ grid of prisms.
This simulation totalized 800 prisms whose thicknesses are 50 m.
The 400 deepest prisms are centered at 825 deep, with pressure contrast equal to $-20$ MPa and the 400 shalowest prisms are centered at 775 deep, with pressure contrast equal to $-40$ MPa  

We calculate the displacement fields due to the pressure change in the whole cylindrical reservoir under non uniform depletion.
Figure \ref{fig:displacement_non_uniform_depletion} shows cross-sections at $x  = 0$ m of the horizontal and vertical displacements, in 2D contour plots, calculated in the whole reservoir by using our methodology.
Figure \ref{fig:displacement_z_levels_non_uniform_depletion} shows the $x-$component displacement and vertical displacement that are calculated by our methodology along the 
$x-$axis, at $y = 0$ m and considering four surfaces located at the following depths:  seafloor ($z = 0$ m), reservoir top ($z = 750$ m), reservoir center ($z = 800$ m) and reservoir bottom ($z = 850$ m).

By comparing Figure \ref{fig:displacement_non_uniform_depletion} with Figures \ref{fig:displacement}, we can note similar behaviours of the displacement fields.
However,  the displacement fields of reservoir under non uniform depletion (Figure \ref{fig:displacement_non_uniform_depletion}) attain higher values because the higher variation of the pressure in the whole cylindrical reservoir.

In general, the displacements on the seafloor, the top, the center and the bottom of the reservoir under non uniform depletion 
(Figure \ref{fig:displacement_z_levels_non_uniform_depletion}) show similar behaviors to the corresponding surfaces of the reservoir under uniform depletion 
(Figure \ref{fig:displacement_z_levels}).
However, we can note higher displacements of  these surfaces in the reservoir under non uniform depletion (Figure \ref{fig:displacement_z_levels_non_uniform_depletion}) due to the higher variation of the pressure in the whole cylindrical reservoir.
Moreover, we can observe that the $x-$component displacements
(Figure \ref{fig:displacement_z_levels_non_uniform_depletion}a)
produced by the top (black line) and the bottom (red line) of the reservoir under non uniform depletion are not coincident to each other and the vertical displacement on the center of the reservoir under non uniform depletion (blue line in (Figure \ref{fig:displacement_z_levels_non_uniform_depletion}b) varies along the $x-$axis.
Finally, we stress that the subsidence on the seafloor due to the reservoir under non uniform depletion 
(green line in Figure  \ref{fig:displacement_z_levels_non_uniform_depletion}b) attains higher values (close to 20 cm) than the subsidence of the seafloor due to the reservoir under uniform depletion (green line in Figure  \ref{fig:displacement_z_levels}b).

Figure \ref{fig:Null_stress_non_uniform_depletion} shows the  null stress through the free surface at the plane $z=0$ m (equations \ref{eq:null_stress} and \ref{eq:null_stress_i}) due to reservoir under a non uniform depletion.

\begin{figure}[h]
\includegraphics[width=8.3cm]{Fig/Figure_Displacement_non_uniform_depletion.png}
\caption{Reservoir under non uniform depletion: (a) Horizontal displacement (equation \ref{eq:horizontal_displacement}) and (b) vertical displacement (equation \ref{eq:u_til_z}) by our methodology that uses the closed expressions of the volume integrations given by \cite{Nagyetal2000} and \cite{Nagyetal2002} (equations \ref{dx1}-\ref{dz2})}
\label{fig:displacement_non_uniform_depletion}
\end{figure}

\begin{figure}[h]
\includegraphics[width=8.3cm]{Fig/Figure_Displacement_z_levels_non_uniform_depletion.png}
\caption{Reservoir under non uniform depletion: (a) Horizontal x-component displacement and (b) vertical displacement by our methodology that uses the closed expressions of the volume integrations given by \cite{Nagyetal2000} and \cite{Nagyetal2002} (equations \ref{dx1}-\ref{dz2}).
These displacements are calculated along the x-axis, at $y = 0$ m and $z$ located at the depths of:  seafloor ($z = 0$ m), reservoir top ($z = 750$ m), reservoir center ($z = 800$ m) and reservoir bottom ($z = 850$ m).}
\label{fig:displacement_z_levels_non_uniform_depletion}
\end{figure}


\begin{figure*}[h]
\includegraphics[width=12cm]{Fig/Figure_Null_stress_non_uniform_depletion.png}
\caption{Reservoir under non uniform depletion: (a) $x-$, (b) $y-$, and (c) $z-$ components of the stress at the free surface.}
\label{fig:Null_stress_non_uniform_depletion}
\end{figure*}


\subsubsection{Reservoir with arbitrary geometry and under arbitrary pressure changes}

The reservoir model is a simplification of a realistic reservoir 
located in a production oil field in offshore Brazil.
The entire model comprises dimensions of 14 km in the north-axis, 13 km in the
east-axis, and 0.6 km in the down-axis. 
The depths to the top and bottom of the reservoir model are 2,712 m and 3,312 m, respectively. 
The components of the displacements are calculated at 0 m deep, 
on a regular grid of 100 $\times$ 80  observation points, with a grid spacing of XXX and XXX m along the north- and east-directions, respectively. 

We  discretized the reservoir  along the $x-$, $y-$ and $z-$ directions into an $14 \times 13 \times 2$ grid of prisms.
The Young’s modulus is  3300 (in MPa), the Poisson's coefficient is 0.25,
the uniaxial compaction coefficient $C_{m}$  (equation \ref{eq:Cm}) is $2.2525 \: 10^{-4}$
$\textrm{ MPa}^{-1}$.
Figure \ref{fig:pressure_complex_reservoir} shows a 3D perspective view of the pore pressure distribution of the simulated reservoir. 
We can see that pressures vary from 0 to 35.6 MP

Figure \ref{fig:displacement_complex_reservoir} shows cross-sections at $x  = 8$ km of the horizontal and vertical displacements, in 2D contour plots, calculated in the whole reservoir by using our methodology.
Figure \ref{fig:Null_stress_complex_reservoir} shows the  null stress through the free surface (equations \ref{eq:null_stress} and \ref{eq:null_stress_i}) due to reservoir with arbitrary geometry and under arbitrary pressure disribution.

\begin{figure}[h]
\includegraphics[width=8.3cm]{Fig/Figure_Pressure_complex_reservoir.png}
\caption{Reservoir with arbitrary geometry and under arbitrary pressure changes: 3D perspective view of the pore pressure distribution of a realistic reservoir 
located in a production oil field in offshore Brazil.}
\label{fig:pressure_complex_reservoir}
\end{figure}

\begin{figure}[h]
\includegraphics[width=8.3cm]{Fig/Figure_Displacement_complex_reservoir.png}
\caption{Reservoir with arbitrary geometry and under arbitrary pressure changes: (a) Horizontal displacement (equation \ref{eq:horizontal_displacement}) and (b) vertical displacement (equation \ref{eq:u_til_z}) by our methodology that uses the closed expressions of the volume integrations given by \cite{Nagyetal2000} and \cite{Nagyetal2002} (equations \ref{dx1}-\ref{dz2})}
\label{fig:displacement_complex_reservoir}
\end{figure}

\begin{figure*}[h]
\includegraphics[width=12cm]{Fig/Figure_Null_stress_complex_reservoir.png}
\caption{Reservoir with arbitrary geometry and under arbitrary pressure changes: (a) $x-$, (b) $y-$, and (c) $z-$ components of the stress at the free surface.}
\label{fig:Null_stress_complex_reservoir}
\end{figure*}


\conclusions  %% \conclusions[modified heading if necessary]
TEXT
Parei neste texto ****

%% The following commands are for the statements about the availability of data sets and/or software code corresponding to the manuscript.
%% It is strongly recommended to make use of these sections in case data sets and/or software code have been part of your research the article is based on.

\codeavailability{TEXT} %% use this section when having only software code available


\dataavailability{TEXT} %% use this section when having only data sets available


\codedataavailability{TEXT} %% use this section when having data sets and software code available


\sampleavailability{TEXT} %% use this section when having geoscientific samples available


\videosupplement{TEXT} %% use this section when having video supplements available


%\appendix
%\section{}    %% Appendix A

%\subsection{}     %% Appendix A1, A2, etc.


%\noappendix       %% use this to mark the end of the appendix section. Otherwise the figures might be numbered incorrectly (e.g. 10 instead of 1).

%% Regarding figures and tables in appendices, the following two options are possible depending on your general handling of figures and tables in the manuscript environment:

%% Option 1: If you sorted all figures and tables into the sections of the text, please also sort the appendix figures and appendix tables into the respective appendix sections.
%% They will be correctly named automatically.

%% Option 2: If you put all figures after the reference list, please insert appendix tables and figures after the normal tables and figures.
%% To rename them correctly to A1, A2, etc., please add the following commands in front of them:

%\appendixfigures  %% needs to be added in front of appendix figures

%\appendixtables   %% needs to be added in front of appendix tables

%% Please add \clearpage between each table and/or figure. Further guidelines on figures and tables can be found below.



\authorcontribution{TEXT} %% this section is mandatory

\competinginterests{TEXT} %% this section is mandatory even if you declare that no competing interests are present

\disclaimer{TEXT} %% optional section

\begin{acknowledgements}
TEXT
\end{acknowledgements}

%% REFERENCES

%% The reference list is compiled as follows:

\begin{thebibliography}{}


\bibitem[Beltrami (1902–1920)]{Beltrami}
Beltrami,  E.: Opere Matematiche, Hoepli, Milano, 1902–1920.

\bibitem[Borges et al. (2020)]{Borges2020}
Borges, F., Landro M., and  Duffaut, K.: Time-lapse seismic analysis of overburden water injection at the Ekofisk field, southern North Sea, GEOPHYSICS, 85, B9–B21, 2020

\bibitem[Dahm et al. (2015)]{Dahmetal15}
Dahm, T.,  Cesca, S., Hainzl, S., Braun,  T., and Krüger, F.: Discrimination
between induced, triggered, and natural earthquakes close to hydrocarbon reservoirs: A probabilistic approach based on the modeling of depletion-induced stress changes
and seismological source parameters, J. Geophys. Res. Solid Earth, 120,
2491–2509, 2015.

\bibitem[Kellogg (1929)]{Kellogg29} 
Kellogg, O. D.: Foundations of Potential Theory, rederick Ungar Publishing Company, 1929.

\bibitem[Fukushima (2020)]{Fukushima2020} 
Fukushima, T.: Speed and accuracy improvements in standard algorithm for prismatic gravitational field, Geophys. J. Int.,  222, 1898–1908,  2020.


\bibitem[Geertsma (1957)]{Geertsma57} 
Geertsma, J.: A remark on the analogy between thermoelasticity and the
elasticity of saturated porous media. J. Mech. Phys. Solids, 6, 13–16, 1957

\bibitem[Geertsma (1966)]{Geertsma66} 
Geertsma, J.: Problems of rock mechanics in petroleum production engineering, in: Proc. First Cong. Int. Soc. Rock Mech., 1, 585–594, 1966.

\bibitem[Geertsma (1973)]{Geertsma73} 
Geertsma, J.: Land Subsidence above compacting oil and gas reservoirs, J. Pet. Tech. 25, 734-744, 1973.

\bibitem[Geertsma and van Opstal (1973)]{Geertsma&Opstal73} 
Geertsma, J., and van Opstal, G.: A numerical technique for predicting subsidence above compacting reservoirs, based on the nucleus of strain concept,” Verhandelingen Kon.
Ned. Geol. Mijnbouwk., 28, 63–78, 1973.

\bibitem[Grigoli et al. (2017)]{Grigolietal17}
Grigoli, F., Cesca, S., Priolo. E., Rinaldi, A.P., Clinton, J.F., Stabile, T.A., Dost, B., Fernandez, M.G., Wiemer, S. and, Dahm T.: Current challenges in monitoring, discrimination, and management of induced seismicity related to underground industrial activities: a European perspective, Rev, Geophys., 55, 310–340, 2017
 
\bibitem[Goodier (1937)]{Goodier37}
Goodier, J. N.: On the Integration of the Thermo-elasti equations, Phil. Mag. 7, 1017-1032, 1937. 

\bibitem[Mehrabian and  Abousleiman (2015)]{Mehrabian&Abousleiman15}
Mehrabian, A., and  Abousleiman,  Y.A.: Geertsma’s subsidence solution extended to layered stratigraphy, J. Pet. Sci. Eng., 130, 68–76, 2015.

\bibitem[Mindlin and Cheng (1950)]{Mindlin&Cheng50}
Mindlin, R.D.  and   Cheng, D.H.: Thermoelastic stress in the semi-infinite solid, J. Appl. Phys. 21, 931–933, 1950.

\bibitem[Sen (1951)]{Sen51}
Sen, B.: Note on the Stresses Produced by Nuclei of Thermoelastic Strain
in a Semi-infinite Solid, Quart. Appl. Math., 8, 365–369, 1951. 


\bibitem[Nagy et al. (2000)]{Nagyetal2000}
Nagy, D.,  Papp, G., and  Benedek, J.: The gravitational potential and its
derivatives for the prism: Journal of Geodesy, 74, 552–560, 2000.


\bibitem[Nagy et al. (2002)]{Nagyetal2002}
Nagy, D.,  Papp, G., and  Benedek, J.: Corrections to "The gravitational potential and its
derivatives for the prism": Journal of Geodesy, 76, 475, 2002.


\bibitem[Muñoz and Roehl (2017)]{Munoz&Roehl17}
Muñoz, L.F.P. and Roehl, D: An analytical solution for displacements due to reservoir
compaction under arbitrary pressure changes, Appl. Math. Modell., 52, 145–159, 2017.

\bibitem[Tao (1971)]{Tao71}
Tao, L.N: Integration of the dynamic thermoelastic equations, 
Int. J. Eng. Sci., 9, 489–505, 1971

\bibitem[Tempone et al. (2010)]{Tempone10}
Tempone, P.,  Fjaer, E., and  Landro, M.: Improved solution of displacements due to a compacting reservoir over a rigid basement, Appl. Math. Modell. 34, 3352–3362, 2010.

\bibitem[Tempone et al. (2012)]{Tempone12}
Tempone, P.,  Landro, M., and Fjaer, E.: I4D gravity response of compacting reservoirs: Analytical approach, Geophysics, 77, G45–G54, 2012.

\bibitem[Segall (1992)]{Segall92}
Segall, P.: Induced stresses due to fluid extraction from
axisymmetric reservoirs. Pure Appl. Phys. 139, 536–560, 1992.

\bibitem[Sharma (1956)]{Sharma56}
Sharma, B.D.: Stresses in an infinite slab due to a nucleus of thermoelastic strain in it, Z. Angew. Math. Mech., 36, 565–589, 1956. 

\bibitem[van Thienen-visser and Fokker (2017)]{vanThienen-visser&Fokker17}
van Thienen-Visser, K. and Fokker, P. A.:The future of subsidence modelling: compaction and subsidence due to gas depletion of the Groningen gas field in the Netherlands, 96, Netherlands Journal of Geosciences, 96, s105–s116, 2017.

\end{thebibliography}

%% Since the Copernicus LaTeX package includes the BibTeX style file copernicus.bst,
%% authors experienced with BibTeX only have to include the following two lines:
%%
%% \bibliographystyle{copernicus}
%% \bibliography{example.bib}
%%
%% URLs and DOIs can be entered in your BibTeX file as:
%%
%% URL = {http://www.xyz.org/~jones/idx_g.htm}
%% DOI = {10.5194/xyz}


%% LITERATURE CITATIONS
%%
%% command                        & example result
%% \citet{jones90}|               & Jones et al. (1990)
%% \citep{jones90}|               & (Jones et al., 1990)
%% \citep{jones90,jones93}|       & (Jones et al., 1990, 1993)
%% \citep[p.~32]{jones90}|        & (Jones et al., 1990, p.~32)
%% \citep[e.g.,][]{jones90}|      & (e.g., Jones et al., 1990)
%% \citep[e.g.,][p.~32]{jones90}| & (e.g., Jones et al., 1990, p.~32)
%% \citeauthor{jones90}|          & Jones et al.
%% \citeyear{jones90}|            & 1990


% Figures

%% FIGURES

%% When figures and tables are placed at the end of the MS (article in one-column style), please add \clearpage
%% between bibliography and first table and/or figure as well as between each table and/or figure.

% The figure files should be labelled correctly with Arabic numerals (e.g. fig01.jpg, fig02.png).


%% ONE-COLUMN FIGURES

%%f
%\begin{figure}[t]
%\includegraphics[width=8.3cm]{FILE NAME}
%\caption{TEXT}
%\end{figure}
%
%%% TWO-COLUMN FIGURES
%
%%f
%\begin{figure*}[t]
%\includegraphics[width=12cm]{FILE NAME}
%\caption{TEXT}
%\end{figure*}
%
%
%%% TABLES
%%%
%%% The different columns must be seperated with a & command and should
%%% end with \\ to identify the column brake.
%
%%% ONE-COLUMN TABLE
%
%%t
%\begin{table}[t]
%\caption{TEXT}
%\begin{tabular}{column = lcr}
%\tophline
%
%\middlehline
%
%\bottomhline
%\end{tabular}
%\belowtable{} % Table Footnotes
%\end{table}
%
%%% TWO-COLUMN TABLE
%
%%t
%\begin{table*}[t]
%\caption{TEXT}
%\begin{tabular}{column = lcr}
%\tophline
%
%\middlehline
%
%\bottomhline
%\end{tabular}
%\belowtable{} % Table Footnotes
%\end{table*}
%
%%% LANDSCAPE TABLE
%
%%t
%\begin{sidewaystable*}[t]
%\caption{TEXT}
%\begin{tabular}{column = lcr}
%\tophline
%
%\middlehline
%
%\bottomhline
%\end{tabular}
%\belowtable{} % Table Footnotes
%\end{sidewaystable*}
%
%
%%% MATHEMATICAL EXPRESSIONS
%
%%% All papers typeset by Copernicus Publications follow the math typesetting regulations
%%% given by the IUPAC Green Book (IUPAC: Quantities, Units and Symbols in Physical Chemistry,
%%% 2nd Edn., Blackwell Science, available at: http://old.iupac.org/publications/books/gbook/green_book_2ed.pdf, 1993).
%%%
%%% Physical quantities/variables are typeset in italic font (t for time, T for Temperature)
%%% Indices which are not defined are typeset in italic font (x, y, z, a, b, c)
%%% Items/objects which are defined are typeset in roman font (Car A, Car B)
%%% Descriptions/specifications which are defined by itself are typeset in roman font (abs, rel, ref, tot, net, ice)
%%% Abbreviations from 2 letters are typeset in roman font (RH, LAI)
%%% Vectors are identified in bold italic font using \vec{x}
%%% Matrices are identified in bold roman font
%%% Multiplication signs are typeset using the LaTeX commands \times (for vector products, grids, and exponential notations) or \cdot
%%% The character * should not be applied as mutliplication sign
%
%
%%% EQUATIONS
%
%%% Single-row equation
%
%\begin{equation}
%
%\end{equation}
%
%%% Multiline equation
%
%\begin{align}
%& 3 + 5 = 8\\
%& 3 + 5 = 8\\
%& 3 + 5 = 8
%\end{align}
%
%
%%% MATRICES
%
%\begin{matrix}
%x & y & z\\
%x & y & z\\
%x & y & z\\
%\end{matrix}
%
%
%%% ALGORITHM
%
%\begin{algorithm}
%\caption{...}
%\label{a1}
%\begin{algorithmic}
%...
%\end{algorithmic}
%\end{algorithm}
%
%
%%% CHEMICAL FORMULAS AND REACTIONS
%
%%% For formulas embedded in the text, please use \chem{}
%
%%% The reaction environment creates labels including the letter R, i.e. (R1), (R2), etc.
%
%\begin{reaction}
%%% \rightarrow should be used for normal (one-way) chemical reactions
%%% \rightleftharpoons should be used for equilibria
%%% \leftrightarrow should be used for resonance structures
%\end{reaction}
%
%
%%% PHYSICAL UNITS
%%%
%%% Please use \unit{} and apply the exponential notation


\end{document}
